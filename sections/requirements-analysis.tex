\section{Requirements Analysis}
\label{section:requirements-analysis}

Requirements analysis was performed on the basis of previously collected use cases.
Required features and parts of the existing software that
had to be refactored were identified.

\subsection{Use Cases}

    \subsubsection{Accessing Students' Self-Regulated Learning Behaviour}
    In the context of TLA, the survey tool will be used to access students' self-regulated
    learning behaviour. Several psychometrical instruments exist for this, but
    for the purpose of this thesis only instruments based on or resembling
    the motivated strategies for learning questionnaire (MSLQ) were considered.
    One such instrument described in the 1994 paper 
    ``Lernstrategien im Studium : Ergebnisse zur Faktorenstruktur und Reliabilitat eines neuen Fragebogens'' \cite{lernstrategien-wild-schiefele}
    in particular was used as a guide to define new requirements for the platform.
     
     \subsubsection{Embedding in Moodle}
     \label{use-case:embedding}
     Moodle is an LMS used at Goethe University. The platform supports embedding of external
     tools inside of a course context. Because previous and simultaneous efforts by Prof. Drachsler's
     work group and collaborators already target Moodle, the new survey tool should also integrate
     with it. Surveys are still created using the survey tool's own user interface.
     Once a survey is created, it may be embedded into Moodle's course context via the LTI
     protocol. Data subjects already using the LMS may then directly participate in
     the survey without leaving the website.
 
     \subsubsection{Stand-Alone Survey Platform}
     In addition to directly embedding the survey tool into an LMS, it should be possible
     for a data subject to participate in a survey, if no LMS is used.
     This use case was carried over from the previous version.

     \subsubsection{Data Provider as Part of the TLA Ecosystem}
     As metnioned before, the survey tool will contribute to the TLA ecosystem
     by providing data to TLA's facts engine, using xAPI to communicate
     information about responses and events on the platform.

\subsection{Refactoring \& Restructuring}
\subsubsection{Database Abstraction}
The previous version of the survey tool uses a self-authored database abstraction library,
the object-document-mapper (ODM). This library is capable of persisting python objects
and relationships between them. It does this by serializing objects to JSON and storing them
using MongoDB. Runtime interactions with these objects are intercepted and converted into 
database queries using the descriptor pattern (https://docs.python.org/3.7/howto/descriptor.html).
This approach worked well for the limited use case, but it proved difficult to implement
a transaction model that follows ACID properties.
To achieve transaction safety without compromising API transparency, an existing
object-relational-mapper (ORM) was chosen to replace the ODM. Key factors for the choice
of ORM were:

\begin{description}
    \item[Maturity of the project.] Data integrity is critical to the application. There is an
    increasing chance of bugs being found and resolved with increasing project age.
    \item[Support for inheritance mapping.] Not being able to effectively use class hierarchies 
    had proven itself to be a major burden on development speed during development of the previous
    version. Attributes of related classes had to be explicitely duplicated. 
    This was not only confusing to read but also difficult to maintain. 
    \item[Support for JSON or hash map columns.] JSON columns provide a convenient way to store multiple 
    translations of a text column, without producing additional joins between the main table and
    supplemental translation tables. Storing multiple representations of a string
    inside a relational database usually involves a 1 to n relationship between
    the internationalised entity and it's translations. With the use of JSON columns,
    multiple translations of a string can be stored as a dictionary inside a single
    column of the entity itself.
\end{description}

In table \ref{table:orm-feature-comparison} a comparison between the some of the most
popular ORM libraries available for Python 3 is presented. 
Because of the project age, previous experience with the library and it's reliability 
in production environments and the large feature set, SQLAlchemy was chosen to replace the ODM.

\begin{figure}
 \centering
 \begin{tabularx}{\textwidth}{|l|X|X|X|X|}
     \hline
     \diagbox{Feature}{ORM} & SQLAlchemy & PeeWee & PonyORM & SQLObject \\
     \hline
     Declarative API & Yes & Yes & Yes & Yes \\
     Eager loading of relationships & Configurable & No & Configurable & No \\
     Lazy loading of relationships & Configurable & Yes & Configurable & Yes \\
     Query caching & Yes & Yes & Yes & Yes \\
     Cascades & update, delete & update, delete & delete & delete \\
     Inheritance mapping & Yes & No & Yes & No \\
     JSON columns & Using PostgreSQL & No & Yes & Using PostgreSQL \\
     First released & 2005 & 2010 & 2014 & 2003 \\
     \hline
 \end{tabularx}
 \caption{Comparison of python ORM libraries}
 \label{table:orm-feature-comparison}
\end{figure} 

\subsection{Survey Content}
 To allow other surveys apart from the EFLA survey to work well with the platform, 
 the following requirements were established:

 \begin{description}
     \item[Response ranges should be adjustable.] Responses will still be on
     a discrete scale with a step size of 1. Start and end of the scale should
     be adjustable.
     \item[Labels for the response ranges should be adjustable.] The EFLA survey
     uses the phrases ``Strongly disagree'' and ``Strongly agree'' to describe
     the lower and upper ends of the response scale respectively. Other surveys
     may use different descriptions, hence these descriptions should be adjustable.
     \item[Question order should be configurable and optionally randomizable.]
 \end{description}

\subsection{Template Management}
The set of questionnaires which are of interest is changing over time.
Initially, the MSLQ was considered to be the main questionnaire of interest
for accessing self regulated learning behaviour. Because of the questionnaire's
size, alternatives were researched by Prof. H. Drachsler and associates.
As new research is published and questionnaires become better statistically validated, the number of survey
 items needed to access a certain research question may decrease.
 For this reason, more current questionnaires may be preferred over
 older ones. It should be easy to integrate new survey items in the future.
 At the same time, once a questionnaire is reasonably well verified, multiple users
 may be interested in using the same questionnaire for different groups of participants.
 The previous version has rudimentary template support, but templates are supplied as static
 YAML files on build and can not be updated from the user interface.
 To overcome this limitation, the following requirements were established:

 \begin{description}
     \item[Templates should be user-contributed.] Data clients with special
     privileges, called contributors, should be able to publish new templates
     using just the user interface.
     \item[Individual survey items may be templates.] To create slightly modified
     versions of a questionnaire, it is useful to re-use existing survey items and not
     just entire questionnaires.
     \item[Templates should be modifiable after creation.] Otherwise, editing
     errors would cause the template to become unusable and the entire template
     has to be created again.
     \item[Changes should be traceable.] When the maintainer of a template 
     modifies the template, other users who own copies of it should be able to 
     review the changes.
 \end{description}

\subsection{Support for xAPI}
 The previous version only supports limited data analysis capabilities.
 The new version should interface directly with the TLA 
 infrastructure to allow for further data processing by TLA's analysis engine. 
 TLA uses the xAPI statement format as it's common data representation. To provide data
 using xAPI to TLA, the following requirements were established:

 \begin{description}
     \item[Survey results should be published to an LRS via xAPI.] A suiting xAPI 
     representation of survey results has to be defined and should be transmitted
     to the LRS vie HTTP as defined by the xAPI specification \cite{xapi-spec}.
     \item[The destination LRS should be configurable.] This allows the location
     of TLA to change in the future. It also decouples the survey tool from TLA 
     and allows interoperability with other LRSs.
     \item[xAPI statements must not be lost due to failure.] When publishing data
     to an external storage, data integrity is no longer just determined by
     database integrity. Appropriate measures must be taken to ensure 
     re-transmission in case of network failure. If transmission is not possible
     due to misconfiguration, an offline fallback method has to be provided.
 \end{description}

\subsection{Embedding \& LTI Launch}
 To achieve compatibility between Moodle and the survey tool described in \ref{use-case:embedding}, 
 the following requirements were established:

 \begin{description}
     \item[The survey tool has to implement the LTI launch protocol.] 
     The LTI launch protocol is used to embed and launch external tools 
     by the Moodle LMS. The survey tool has to recognize LTI launch requests
     and respond with an embeddable version of the survey.
     \item[Data subjects should not have to authenticate.] When using the survey tool
     from within Moodle, data subjects have already authorized with CAS in order
     to log in to the LMS. Requiring another form of user authentication after
     this point would break seamless user experience.
 \end{description}

 LTI uses OAuth 1 to sign requests. This provides a way for the tool provider
 to verify the identity of the LMS and the authenticity of the LTI request.
 It was explicitely not required to implement OAuth as part of this work.

\subsection{Privacy Considerations}
 Since May 25th 2018, the general data protection regulation applies to members of the EU \cite{gdpr-info}.
 The author is not in any way qualified to provide legal interpretation on the regulation.
 However, in consultation with Prof. H. Drachsler the following requirements
 were identified to conform with the regulation:

 \begin{description}
     \item[Data deletion for data subjects.] Some personal data has to be stored
     on the server in order to identify data subjects between requests. Because of the
     right to be forgotten, there has to be a way to delete personal data stored for
     a specific data subject. It is sufficient if this is not automated and can be performed
     by the site's administrator.
     \item[No user accounts are available to external data clients.] If third parties are allowed to use
     the service, it can not be guaranteed that they will follow GDPR guidelines.
     To avoid responsibility for third parties' actions on the site, there will
     be no way to sign up as a data client that does not involve contacting the administrator.
 \end{description}
 
 Other rights covered by the GDPR include the right to view and export personal data.
 Since all personal data collected is published to the TLA infrastructure,
 this will be implemented as part of the TLA infrastructure and is not
 part of the survey tool.
