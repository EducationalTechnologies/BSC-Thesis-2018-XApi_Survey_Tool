\section{Outlook}
	\subsection{Beta-Testing and Productive Use}
		The survey tool will be used by participants of the Educational Technologies
		Seminar held by Prof. H. Drachsler in the 2018/19 winter semester for the first time.
		In the seminar, students may choose to participate in a test-run
		of the TLA infrastructure. In this test-run, data is gathered
		from different sources, including the survey tool, to test
		the capabilities of the TLA infrastructure and to discover
		potential issues with the system. Participants will
		take a survey accessing motivation and self-regulation
		with regards to their learning behavior. This data will
		then be combined with activity data collected from Moodle,
		regularly self-reported data on goal setting and time management
		and physical attendance times.

	\subsection{Future Features}
		There are many additional features which could be added to
		the existing codebase, once the known issues have been resolved.
		This section describes some of these features and gives a
		brief outline as to how they could be integrated.

		\subsubsection{Configurable Authentication Mechanisms for xAPI Consumers}
			Some xAPI consumers only accept xAPI statements from known
			sources and thus require authentication before statements
			can be transmitted. The xapi-publisher service already supports
			this and bundles statements by destination and authentication mechanism, 
			so that sessions may be re-used. This is not
			described in this thesis, as this was added as part of a hotfix
			in the last days of this thesis. The user interface
			and backend service do not yet support configuration
			by the user for this feature. In the future, the tool
			could provide a set of common authentication mechanisms
			for the data client to choose from. Depending on the
			chosen mechanism, different configuration forms
			would then be displayed in the user interface to supply
			credentials for the mechanism. 

		\subsubsection{Additional Response Types}
			In addition to questions which can be answered on a numerical
			scale, other kinds of response types could be added to the survey
			tool. This is particularly interesting for use cases outside
			psychometry, say, when performing course evaluations.
			Examples for additional response types are free text,
			boolean (yes or no), date \& time and date- \& time-ranges.
			Integrating these kinds of questions would mainly involve
			subclassing the \inline{Question} class and designing
			new user interface components. The overall survey logic
			should still work with these new types. xAPI also already
			supports these kinds of answer values.

		\subsubsection{Additional Formatting Options for Data Clients}
			At the moment, questions present a question text, as
			well as labels for the upper and lower bounds of the response
			range to the data subject. The question text does not
			support formatting. This format is not suited to
			longer paragraphs of text, say, when context has
			to be provided for the question. In the future, questions
			could support formatting using a markup language
			like markdown or textile. The benefit of using such a markup
			language is that no changes have to be made to the way
			question texts are stored, as the markup
			can be stored as plain text. In addition, good library support
			exists for converting these markup languages to HTML
			in a safe manner. In the user interface, the markup
			could either be entered by hand, if the markup language
			is not too complicated, or an already existing
			JavaScript markdown editor like SimpleMDE could be integrated.
			Depending on the markup, this would also allow for images
			and videos to be embedded into the survey, which
			could greatly expand the set of surveys that can
			be realized with the survey tool.

		\subsubsection{A/B Questionnaires}
			Some surveys study the effect of survey taking itself
			on it's subjects. For this reason, different versions
			of the same questionnaire might be presented to
			data subjects. One such example would be a survey,
			where one version uses an even number of response
			values and the other uses an odd number. Currently,
			the entire questionnaire would have to be duplicated in
			order to achieve this. To solve this, a special
			copy of a questionnaire could be created, which is
			also optionally modifiable. Non-modified parts
			of the questionnaire would then merely reflect the
			state of the original questionnaire, whereas modified
			parts would override the original questionnaire's content.
			Participants could then be randomly assigned one of these
			questionnaires when accessing the survey.

		\subsubsection{Version Control for Templates}
			Instead of immediately propagating changes to templates
			to all copies of it, data clients owning a copy
			could be notified of the change and then choose to
			move to the new version or keep using the old one.
			This feature would involve keeping track of
			content revisions for every survey item.
			Instead of overwriting content when making changes
			to a survey item, the changes would instead be
			applied to a copy of the item, the new revision.
			Copies of templates would then also contain information
			about the revision they are shadowing. 
			Up- or downgrading a survey item to another revision
			would then only involve changing the copy's
			\inline{reference_to} relationship to point to the newer
			or older revision.