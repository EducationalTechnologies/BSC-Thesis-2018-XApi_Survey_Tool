\subsection{Terminology}
\begin{description}
	\item[Block access] 
		An access to a block of secondary storage. Since accessing secondary storage
		presents itself as the bottleneck in database latency, time complexity of
		database operations is often represented by the number of block accesses rather
		than the number of operations in RAM.
    \item[data subject]
    \item[data client]
    \item[survey item] 
    \item[Tool provider]
	    In the context of LTI, the term \textit{tool provider} is used to describe a
	    system which provides an external tool to an LMS, extending the LMS's capabilities.
    \item[Learning record store]
    	A database system, possibly including analysis and visualisation capabilities, 
	    for storing data of interest to learning analytics applications.
	    In the context of this thesis, LRS refers to a system for storing and analysing xAPI statements in particular.
    	Examples for LRS systems are the TLA facts engine an HT2Labs's Learning Locker \ref{ht2labs-learninglocker}.
    \item[Learning management system]
	    A content management system, specifically designed for e-learning applications.
	    Examples for LMSs are Moodle and OLAT.
\end{description}